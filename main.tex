\documentclass[USenglish,oneside,twocolumn]{article}
\usepackage[utf8]{inputenc}
\usepackage[table,xcdraw]{xcolor}
\usepackage[big]{dgruyter_NEW}
\usepackage{multirow}
\usepackage{subcaption}

\usepackage{rotating}
\usepackage{xcolor}
\date{June 2019}

\usepackage{paralist}
\usepackage{comment}

\usepackage{natbib}
\usepackage{graphicx}

\newcommand{\fixme}[1]{\textcolor{red}{\textbf{FIXME:} #1}}
\newcommand{\todo}[1]{\textcolor{red}{\textbf{TODO:} #1}}
\newcommand{\new}[1]{\textcolor{blue}{\textbf{NEW:} #1}}


\newif\ifcomment
\commenttrue %commenttrue commentfalse
%\commentfalse
\ifcomment
\newcommand{\guillermo}[1]{{\bf \textcolor{blue}{GST: #1}}}
\newcommand{\figurenote}[1]{{\bf \textcolor{blue}{Note: #1}}}
\newcommand{\michal}[1]{{\bf \textcolor{orange}{MTK: #1}}}
\newcommand{\old}[1]{{\textcolor{red}{\st{#1}}}}
\newcommand{\future}[1]{{\textcolor{gray}{FUTURE: #1}}}
\else
\newcommand{\guillermo}[1]{}
\newcommand{\sergio}[1]{}
\newcommand{\jorge}[1]{}
\newcommand{\michal}[1]{}
\newcommand{\old}[1]{}
\newcommand{\future}[1]{}
\fi

\newcommand{\done}[1]{}


\begin{document}

\begin{comment} % Authors camera ready
\author*[1]{XXX XXX}
\author[2]{XXX XXX}
\affil[1]{YYYYY}
\affil[2]{YYYYY}
\end{comment}


%\title{\huge On the Limitations of Code Stylometry in Underground Forums}
%\runningtitle{On the Limitations of Code Stylometry in Underground Forums}

\title{\huge Study of Malware in GitHub and Underground Forums}
\runningtitle{Towards Improving Code-Stylometry Analysis in Underground Forums}

%We have informed app developers and provided specific recommendations so they can improve their privacy practices.

%%  \keywords{Authorship Attribution, Underground Forums, Language Selection, Code Clone Detection}
%  \classification[PACS]{}
 % \communicated{...}
 % \dedication{...}

  \journalname{}
\DOI{Editor to enter DOI}
  \startpage{1}
  \received{..}
  \revised{..}
  \accepted{..}

  \journalyear{..}
  \journalvolume{..}
  \journalissue{..}
 

\maketitle

\section{Related work}
\label{sec:related}

Two key related works are by Rokon et al.~\cite{rokon2020sourcefinder} and Islam et al.~\cite{islam2020hackerscope}. The first presents a method for identifying malware source code repositories on the popular collaborative code platform GitHub, showing that there is a substantial amount (over 7000) malware repositories on the platform, with the number doubling every three years. This presents an unprecedented amount of malware source code available to the researcher. The second work presents a method they dub HackerScope which utilizes this data to drive an analysis of the code authors both in GitHub repositories and their relation to a select few internet hacking communities. They focus on the influence factor of the authors and explore how the authors relate to each other and how the malware community as a whole is split into sub-communities in the malware ecosystem, in particular on GitHub. However, they leave the code itself unexplored in favor of analysis of the repositories' metadata, thus leaving a number of interesting research questions unanswered. One such issue is the presence of evolutionary relationships between malware samples. There have been efforts to identify malware lineage in the literature, however mainly due to scarcity of source code samples this has been done on compiled samples~\cite{haq2018malware,ming2015memoized,jang2013towards,suarez2014dendroid}. To our knowledge there are no techniques in the literature for deriving malware lineage from source code. \michal{I haven't been able to find anything. But that can't be right?} \guillermo{Keep digging.}
Thus it remains an open question how we can utilize source code in order to conjure a picture of relationships between malware samples.
\guillermo{What is the benefit of looking at lineage in source code? Two possible answers we need to study/position: i) more granularity, and ii) it can be enriched with meta-data (relevant to attribution in the case of a repository).}


Drawing these relationships between samples requires a way to measure similarity between samples: we draw upon the literature in the software plagiarism detection domain in this respect. There is a plethora of tools available for the academic, spanning development from the 1980's to the present day~\cite{novak2019source}. The work by Ragkhitwetsagul et al.~\cite{ragkhitwetsagul2018comparison} evaluates the most pertinent of these approaches. They identify ccfx and JPlag as the most successful tools in detecting code plagiarism.

\bibliographystyle{apalike}
\bibliography{references}


\end{document}